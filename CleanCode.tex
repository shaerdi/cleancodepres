\documentclass[handout]{beamer}
%\documentclass{beamer}
\usepackage[ngerman]{babel}
\usepackage[utf8]{inputenc}
\usepackage[T1]{fontenc}
\usepackage{amsmath,amssymb}
\usepackage{graphicx}
\usepackage{xcolor}
\usepackage{mathptmx} 
\usepackage{outlines} 
\usepackage{listings}
\usepackage{tikz}

\usepackage{multirow}
\usepackage{booktabs}

\usetheme{ntb}

\title[CleanCode]{Clean Code}
\author{Simon Härdi}
\institute{Institute for Computational Engineering}
\date{19. Mai 2016}

\graphicspath{ {pictures/} }

%% general vector style (bold or with arrow)
\newcommand{\ve}[1]{
    \mathbf{#1}
    %\vec{#1}
}

%% common shortcuts
\providecommand{\argmin}{\operatorname*{argmin}} % operatorname makes _{..} appear centered
\providecommand{\argmax}{\operatorname*{argmax}} % operatorname makes _{..} appear centered
\newcommand{\dd}[1]{\,\mathrm{d}#1} % integration: \int f(x) \dd{x}
\newcommand{\EE}{\mathbb{E}}        % expectation value
\newcommand{\RR}{\mathbb{R}}        % set of real numbers
\newcommand{\CC}{\mathbb{C}}        % set of complex numbers
\newcommand{\NN}{\mathbb{N}}        % set of natural numbers
\newcommand{\OO}{\mathcal{O}}       % big O notation (asymptotic complexity)
\newcommand{\TT}{\mathbb{T}}        % time interval

%% Vectors (lowercase letters)
\renewcommand{\a}{\ve{a}}
\renewcommand{\b}{\ve{b}}
\renewcommand{\c}{\ve{c}}
\renewcommand{\d}{\ve{d}}
\newcommand{\e}{\ve{e}}
\newcommand{\f}{\ve{f}}
\newcommand{\g}{\ve{g}}
\newcommand{\h}{\ve{h}}
\renewcommand{\i}{\ve{i}}
\renewcommand{\j}{\ve{j}}
\renewcommand{\k}{\ve{k}}
\renewcommand{\l}{\ve{l}}
\newcommand{\m}{\ve{m}}
\newcommand{\n}{\ve{n}}
\renewcommand{\o}{\ve{o}}
\newcommand{\p}{\ve{p}}
\newcommand{\q}{\ve{q}}
\renewcommand{\r}{\ve{r}}
\newcommand{\s}{\ve{s}}
\renewcommand{\t}{\ve{t}}
\renewcommand{\u}{\ve{u}}
\renewcommand{\v}{\ve{v}}
\newcommand{\w}{\ve{w}}
\newcommand{\x}{\ve{x}}
\newcommand{\y}{\ve{y}}
\newcommand{\z}{\ve{z}}

%% Matrices (uppercase letters)
\newcommand{\A}{\ve{A}}
\newcommand{\B}{\ve{B}}
\newcommand{\C}{\ve{C}}
\newcommand{\D}{\ve{D}}
\newcommand{\E}{\ve{E}}
\newcommand{\F}{\ve{F}}
\newcommand{\G}{\ve{G}}
\renewcommand{\H}{\ve{H}}
\newcommand{\I}{\ve{I}}
\newcommand{\J}{\ve{J}}
\newcommand{\K}{\ve{K}}
\renewcommand{\L}{\ve{L}}
\newcommand{\M}{\ve{M}}
\newcommand{\N}{\ve{N}}
\renewcommand{\O}{\ve{O}}
\renewcommand{\P}{\ve{P}}
\newcommand{\Q}{\ve{Q}}
\newcommand{\R}{\ve{R}}
\renewcommand{\S}{\ve{S}}
\newcommand{\T}{\ve{T}}
\newcommand{\U}{\ve{U}}
\newcommand{\V}{\ve{V}}
\newcommand{\W}{\ve{W}}
\newcommand{\X}{\ve{X}}
\newcommand{\Y}{\ve{Y}}
\newcommand{\Z}{\ve{Z}}

% \codeemph command
\newcommand{\changefont}[3]{\fontfamily{#1}\fontseries{#2}\fontshape{#3}\selectfont}
\newcommand{\codeemph}[1]{{\changefont{pcr}{m}{n}#1}}

\lstset{ %
  language=C++, % choose the language of the code
  %basicstyle=\small\ttfamily, % the size of the fonts that are used for the code
  basicstyle=\footnotesize, % the size of the fonts that are used for the code
  numbers=none, % where to put the line-numbers
  numberstyle=\small\ttfamily\color[rgb]{0.6,0.6,0.6}, % the size of the fonts that are used for the line-numbers
  stepnumber=1, % the step between two line-numbers. If it's 1 each line
  xleftmargin=4mm,
  % will be numbered
  numbersep=5pt, % how far the line-numbers are from the code
  %backgroundcolor=\color{white}, % choose the background color. You must add \usepackage{color}
  showspaces=false, % show spaces adding particular underscores
  showstringspaces=false, % underline spaces within strings
  showtabs=false, % show tabs within strings adding particular underscores
%  frame=l, % adds a frame around the code
  frame=single,
  tabsize=2, % sets default tabsize to 2 spaces
  breaklines=true, % sets automatic line breaking
  breakatwhitespace=false, % sets if automatic breaks should only happen at whitespace
  % also try caption instead of title
  escapeinside={@}{@}, % if you want to add a comment within your code
  morekeywords={*,...}, % if you want to add more keywords to the set
  keywordstyle=\color[rgb]{0,0,1},
  commentstyle=\color[rgb]{0.133,0.545,0.133}\textit,
  stringstyle=\color[rgb]{0.627,0.126,0.941},
}

%% Code
\usepackage{listings}
\lstset{ %
  language=matlab, % choose the language of the code
  basicstyle=\small\ttfamily, % the size of the fonts that are used for the code
  %numbers=left, % where to put the line-numbers
  %numberstyle=\small\ttfamily\color[rgb]{0.6,0.6,0.6}, % the size of the fonts that are used for the line-numbers
  %stepnumber=1, % the step between two line-numbers. If it's 1 each line
  %xleftmargin=4mm,
  %% will be numbered
  %numbersep=5pt, % how far the line-numbers are from the code
  %backgroundcolor=\color{white}, % choose the background color. You must add \usepackage{color}
  %showspaces=false, % show spaces adding particular underscores
  %showstringspaces=false, % underline spaces within strings
  %showtabs=false, % show tabs within strings adding particular underscores
  frame=off, 
  %%frame=single,
  %tabsize=2, % sets default tabsize to 2 spaces
  breaklines=true, % sets automatic line breaking
  keywordstyle=\color[rgb]{0,0,1},
  commentstyle=\color[rgb]{0.133,0.545,0.133}\textit,
  stringstyle=\color[rgb]{0.627,0.126,0.941},
}



\begin{document}

{
\setbeamertemplate{footline}{} % No footer on first slide
\begin{frame}
\maketitle
\end{frame}
}

%\begin{frame}
    %\frametitle{Inhalt}
    %\tableofcontents
%\end{frame}

\section{Einleitung}

\begin{frame}
\frametitle{Einleitung}
\begin{columns}
    \begin{column}{0.6\textwidth}
        \begin{outline}
            \1 Was ist Clean Code?
            \1[]
            \1 Wie schreibt man Clean Code?
        \end{outline}
    \end{column}
    \begin{column}{0.4\textwidth}
        \includegraphics[width=\linewidth]{cleanCodeBook.jpg}
    \end{column}
\end{columns}
\end{frame}

\begin{frame}
    \frametitle{Wieso dieser Vortrag?}
    \pause
    \begin{center}\begin{tikzpicture}
        \definecolor{lightgreen}{rgb}{0.7,1,0.7};
        \draw[draw=none,fill=lightgreen] (-1.5,0) circle[radius=2];
        \draw (-1.5,0) circle[radius=2];
        \node[align=center] at (-2,0) {Komplexe\\ Algorithmen};
        \node (a) at (-3,-2.5) {Eine Kernkompetenz des ICE};
        \draw[->] (a) to[in=190,out=120] (-3,-1.5) ;

        \pause
        \draw[draw=none,fill=lightgreen] (1.5,0) circle[radius=2];
        \node[align=center] at (2,0) {Guter Code};
        \draw (-1.5,0) circle[radius=2];
        \draw (1.5,0) circle[radius=2];

        \pause
        \node (c) at (0,3) {Nicht das Gleiche!};
        \draw[->] (c) to[in=60,out=0] (2.7,1.7) ;
        \draw[->] (c) to[in=120,out=180] (-2.7,1.7) ;

        \pause
        \begin{scope}
            \clip (-1.5,0) circle[radius=2];
            \clip (1.5,0) circle[radius=2];
            \fill[green] (0,0) circle[radius=3];
        \end{scope}

    \end{tikzpicture}\end{center}

\end{frame}
\begin{frame}[fragile]
    \frametitle{Einführungsbeispiel}
    Ignaz soll einen Bug in einem Code fixen. Er konnte den Fehler auf folgende
    Funktion zurückführen:
\begin{lstlisting}
...
result = testPoints(x,y);
...
\end{lstlisting}
Leider hatte der Autor keine Zeit für Kommentare, und Ignaz ist nicht sicher,
was die Funktion eigentlich machen soll. Helft Ignaz herauszufinden,
was die Funktion \codeemph{testPoints} tut und wo der Fehler liegt.
\end{frame}

\begin{frame}
    \frametitle{Resultat des Codes}
    \begin{center}
        \includegraphics[width=0.8\linewidth]{GrafikResultat.png}
    \end{center}
\end{frame}


\begin{frame}
    \frametitle{Grundsatz:}
    \begin{center}
        \huge \bf Code wird öfters gelesen als geschrieben
    \end{center}
    \vspace{2em}\pause
    Also nicht \textit{'Wie schreibe ich schnell Code?'}, \\
    sondern \textit{'Wie schreibe ich schnell \textbf{zu lesenden} Code?'}
\end{frame}

\begin{frame}
    \frametitle{Wann ist Code einfach zu lesen?}
    \begin{center}
        \includegraphics[width=7cm]{wtfm.jpg}
    \end{center}
\end{frame}

\section{Basics}
\begin{frame}
    \tableofcontents
    \begin{tikzpicture}[overlay,
        shift={(current page.south west)},
        x = \paperwidth,
        y = \paperheight,
    ]
\end{tikzpicture}
    \only<2->{
    \begin{tikzpicture}[overlay,
        shift={(current page.south west)},
        x = \paperwidth,
        y = \paperheight,
    ]
    \draw[red] (0.27,0.62) to[out=0,in=180] ++(0.025,-0.1) node[right] (a) {}
    to[out=180,in=0] ++(-0.025,-0.1);
    \draw[->,red] (0.6,0.5) node[right] {Für alle Codes!} 
    to[out=170,in=10] (a);
    \end{tikzpicture}}
    \only<3->{
    \begin{tikzpicture}[overlay,
        shift={(current page.south west)},
        x = \paperwidth,
        y = \paperheight,
    ]
    \draw[red] (0.45,0.36) to[out=0,in=180] ++(0.025,-0.11) node[right] (b) {}
    to[out=180,in=0] ++(-0.025,-0.11);
    \draw[->,red] (0.6,0.2) node[right,align=left] {Für grössere /\\
    Codes (Projekte)} 
    to[out=170,in=10] (b);
    \end{tikzpicture}}
\end{frame}
\begin{frame}
    \tableofcontents[currentsection]
    \begin{tikzpicture}[overlay,
        shift={(current page.south west)},
        x = \paperwidth,
        y = \paperheight,
    ]
\end{tikzpicture}
\end{frame}
\subsection{Namen}
\begin{frame}
    \frametitle{Namen}
        %TODO: Lustiges Bild
    Grundregeln:

    \begin{outline}
        \pause
        \1 Ein guter Variablenname ist besser als ein Kommentar
        \pause
        \1 Mit einer IDE sind lange Namen kein Problem
        \pause
        \1 Keine Abkürzungen, Füllwörter oder Hungarian Naming
    \end{outline}
\end{frame}
\begin{frame}
    \frametitle{Beispiele}
    \begin{center}
        \includegraphics[width=\linewidth]{NamenBeispiele.PNG}
    \end{center}
    \begin{columns}[t]
        \column{0.5\textwidth}
        \begin{center}
            \huge \color{red} Schlecht
        \end{center}
        \column{0.5\textwidth}
        \begin{center}
            \huge \color{green} Gut
        \end{center}
    \end{columns}
\end{frame}

\subsection{Funktionen}
\begin{frame}
    \frametitle{Funktionen}
    Grundregeln:

    \begin{outline}
        \pause
        \1 Kurz
        \pause
        \1 Kürzer!
        \pause
        \1 Nur eine Sache
        \pause
        \1 Möglichst wenig Argumente
    \end{outline}
\end{frame}
\begin{frame}
    \frametitle{Beispiele}
    \begin{columns}[t]
        \column{0.5\textwidth}
        \begin{center}
            \includegraphics[height=4.5cm]{FunktionSchlecht.png}
        \end{center}
        \column{0.5\textwidth}
        \begin{center}
            \includegraphics[width=\linewidth]{FunktionGut.png}
        \end{center}
    \end{columns}
    \begin{columns}[t]
        \column{0.5\textwidth}
        \begin{center}
            \huge \color{red} Schlecht
        \end{center}
        \column{0.5\textwidth}
        \begin{center}
            \huge \color{green} Gut
        \end{center}
    \end{columns}
\end{frame}

\subsection{Kommentare}
\begin{frame}
    \frametitle{Kommentare}
    Grundregeln:

    \begin{outline}
        \pause
        \1 Kommentare sind \emph{nicht} per se gut
        \pause
        \1 Weniger ist mehr
        \pause
        \1 Ein Kommentar sollte immer wohlüberlegt sein
        \pause
        \1 Kein Code auskommentieren
    \end{outline}
\end{frame}
\begin{frame}
    \frametitle{Beispiele}
        \begin{center}
    \includegraphics[width=\textwidth]{pictures/Kommentare1.PNG}

    \pause
    \vspace{2em}
            \huge \color{green} OK
        \end{center}
\end{frame}
\begin{frame}
    \frametitle{Beispiele}
        \begin{center}
    \includegraphics[width=0.5\textwidth]{pictures/Kommentare3.PNG}

    \pause
    \vspace{2em}
            \huge \color{red} Schlecht: Überflüssig
        \end{center}
\end{frame}
\begin{frame}
    \frametitle{Beispiele}
        \begin{center}
    \includegraphics[width=\textwidth]{pictures/Kommentare2.PNG}

    \pause
    \vspace{2em}
            \huge \color{green} OK
        \end{center}
\end{frame}
\begin{frame}
    \frametitle{Beispiele}
        \begin{center}
    \includegraphics[width=0.5\textwidth]{pictures/Kommentare4.PNG}

    \pause
    \vspace{2em}
            \huge \color{red} Schlecht: Wieso auskommentiert?
        \end{center}
\end{frame}
\begin{frame}
    \frametitle{Beispiele}
        \begin{center}
    \includegraphics[width=\textwidth]{pictures/Kommentare5.PNG}

    \pause
    \vspace{2em}
            \huge \color{red} Schlecht: Kommentar falsch
        \end{center}
\end{frame}


\section{Erweitert}
\begin{frame}
    \tableofcontents[currentsection]
\end{frame}
\subsection{Versionierungssystem}
\begin{frame}
    \frametitle{Versionierungsystem}

    Eine Versionskontrolle ist unerlässlich für
    \begin{outline}
        \1 Absicherung gegen Ausfälle
        \1 Zusammenarbeit von mehreren Leuten
        \1 Schnelles Austesten von neuen Funktionen
        \1 ...
    \end{outline} \pause

    \vspace{2em}Welches System?\hspace{3em}
    \pause 
    Egal! (Git, svn, mercurial)

    \vspace{1em}\pause{\color{red} Keine Ordner mit xy\_V01, xy\_V02\ldots}
\end{frame}
\subsection{Tests}
\begin{frame}
    \frametitle{Tests}
    Woher ist bekannt, ob der Code funktioniert?\pause

    \begin{outline}
        \1 Jeder schreibt Tests, aber of nur in der Konsole
        \1 Gehören zum Code dazu
    \end{outline} \pause

    Unit-Tests:
    \begin{outline}
        \1 Alles einzeln Testen
            \2 Jede Funktion
            \2 Möglichst jeden Spezialfall
    \end{outline}
    $\Rightarrow$ Es gibt für (fast) alle Sprachen bereits sehr bequeme
    Frameworks, um solche Tests aufzusetzen und laufen zu lassen.
\end{frame}
\begin{frame}
    \frametitle{Beispiel - Matlab Unit Tests}
    \url{ch.mathworks.com/help/matlab/matlab_prog/write-simple-test-case-with-functions.html}
    \vspace{2em}

    $\Rightarrow$ Matlab Demo
\end{frame}
\subsection{Test Driven Development}
\begin{frame}
    \frametitle{Test Driven Development}
    Vorgehen:

    \begin{outline}
        \1 Tests \emph{vor} der Funktion schreiben
        \1 Ein Test $\rightarrow$ Eine Funktion $\rightarrow$ repeat
    \end{outline}\pause

    Vorteile:

    \begin{outline}
        \1 Führt zu sehr kleinen Funktionen
        \1 Anforderungen an den Code müssen jeweils \emph{vor} dem Schreiben
        definiert werden
        \1 Debuggen wird stark reduziert und viel einfacher
    \end{outline}
\end{frame}
\begin{frame}
    \frametitle{Beispiel}
    %TODO: 
\end{frame}

\subsection{Successive Refinement}
\begin{frame}
    \frametitle{Successive Refinement}
    \begin{outline}
        \1 Code ist \emph{nicht} fertig, sobald er funktioniert
        \1 Kontinuierlich verbessern (Pfadfinder-Regel)
        \1 Dank Tests keine Angst, etwas kaputt zu machen
    \end{outline}
\end{frame}

\section{Fazit}
\begin{frame}
    \begin{tikzpicture}[overlay,
        shift={(current page.south west)},
        x = \paperwidth,
        y = \paperheight,
    ]
\end{tikzpicture}
    \frametitle{Fazit}
    \begin{outline}
        \1 Beschreibende Namen
        \1 Kurze Funktionen
        \1 Kommentare überlegt benutzen
        \1 Versionierungssystem benutzen
        \1 Tests \emph{zuerst} schreiben
    \end{outline}
    \pause
    \begin{tikzpicture}[overlay,
        shift={(current page.south west)},
        x = \paperwidth,
        y = \paperheight,
    ]
    \node[rotate=30,red,align=left] at (0.7,0.2) {Dranbleiben!!};
\end{tikzpicture}
\end{frame}

\begin{frame}
    \frametitle{Übung}
    Ein Programm soll mit dem Sieb des Eratosthenes alle Primzahlen bis zu
    einer gegebenen Zahl bestimmen. Wie könnte dieses Programm aussehen?

    \vspace{1em}\url{https://de.wikipedia.org/wiki/Sieb_des_Eratosthenes}

    \vspace{1em}\pause
    \begin{outline}
        \1 TDD: Welche Tests muss das Programm erfüllen?
        \1 Kurze Funktionen: Hauptprogramm mit grobem Ablauf, jeden Schritt in
        eine Funktion verpacken
        \1 Kommentare: Welche sind wirklich nötig?
    \end{outline}
\end{frame}
\end{document}
